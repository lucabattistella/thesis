

\chapter*{Introduction}
\addcontentsline{toc}{chapter}{Introduction}
Kontsevich's moduli space of stable maps $\M{g}{n}{X}{\beta}$ \cite{KON} is meant to provide a first approximated answer to the question: How many smooth curves of genus $g$ and curve class $\beta$ are there in $X$, that pass through a number of subvarieties $V_1,\ldots,V_n\subseteq X$? Except in the $g=0$, $X$ homogeneous case (and few others), it turns out the geometry of $\M{g}{n}{X}{\beta}$ is overwhelmingly complicated (indeed these spaces satisfy Vakil-Murphy's law in algebraic geometry \cite{VakilMurphy}). Even though its properness makes it well-suited for intersection-theoretic computations, this space is often not endowed with a fundamental class (it may not be equidimensional), but rather with a virtual fundamental class, that enjoys a number of properties expected from an actual fundamental class if there existed one (most notably deformation invariance), but is constructed by means of a rather heavy technical machinery that globalises the (expected) deformation theory of this moduli problem \cites{LiTian,BF}. Even so, the ideal situation of a smooth curve embedding in $X$ is far from generic, both in the actual and virtual sense. Understanding these degenerate contributions from curves of lower genus or degree has been one of the main difficulties since the origins of the subject, in parallel with the equally hard problem of actually computing the Gromov-Witten invariants, for which it has been necessary to introduce clever generating series and to study the remarkable algebraic structures underlying them (quantum cohomology and Frobenius manifolds, see e.g. \cite{ManinFrobenius}), for a direct understanding of the moduli space of maps becomes virtually impossible as soon as the degree gets larger. Comparing Kontsevich's moduli space of stable maps with different modular compactifications of the locus of maps from a smooth curve has proven both a beautiful and a fruitful technique, and it is this approach -in some simple instances- that I am going to discuss within this thesis.

Chapter \ref{ch:qm} deals with the theory of quasimaps. Quasimaps are defined whenever the target belongs to a large class of GIT quotient (when there are no strictly semistable points, the stabilisers are all trvial, and the space acted upon is at most mildly singular; see \cite{CFKM}), in particular when $X$ is a smooth projective toric variety \cite{CF-K}. Quasimaps can be thought of as maps to the quotient stack, such that the preimage of the unstable locus is finite and non-special. They are beloved by the algebraic geometer because their stronger stability condition forces the source curve not to have rational tails, nor contracted rational bridges. In turn, the stability condition can be made to depend on a positive rational parameter $\epsilon$, endowing $\mathbb Q_{>0}$ with a finite wall and chamber structure, such that the $\epsilon\gg0$ condition recovers Kontsevich's stable maps. Elegant wall-crossing formulae for the invariants (or rather their generating series) have been studied in \cites{TodaStableQuotient,CF-K-wallcrossing,CF-K-higher-genus,CF-K-MirrorSymmetry,CladerJandaRuan}. In \S \ref{sec:qm_review} I review and expand on some foundational material in the theory of quasimaps, comparing with well-known results in Gromov-Witten theory, and paying particular attention to the quasimap quantum product - whose relation with Batyrev's quantum ring is not so direct as I hoped in the strictly semi-positive case. In \S \ref{sec:loc_formula} I review the localisation formula for toric quasimaps, which not only is a key ingredient in the proof of wall-crossing formulae for toric targets, but also can be made completely explicit in terms of weights for the big torus action, so that the computation of the invariants in some simple situations becomes a combinatorial game. Finally, in \S \ref{sec:rel_qm} I review the bulk of my joint work with N. Nabijou \cite{BN}, in which we introduce the notion of relative quasimap invariants under the assumptions that $g=0$ and the target $(X|Y)$ is a smooth and very ample pair. We extend Gathmann's formula \cite{Ga} to this setting, relating the virtual classes of the relative quasimap spaces when the tangency requirement is increased by $1$; see Proposition \ref{prop:quasimap_Gathmann_formula}:
\begin{prop*} In the Chow group of $\Q{0}{\alpha}{X|Y}{\beta}$
 \begin{equation*} (\alpha_k \psi_k + x_k^*c_1(L_Y)) \cap \virt{\Q{0}{\alpha}{X|Y}{\beta}} = \virt{\Q{0}{\alpha+e_k}{X|Y}{\beta}} + \virt{\mathcal{D}^Q_{\alpha,k}(X,\beta)} \end{equation*}
where $\mathcal{D}^Q_{\alpha,k}(X,\beta)$ is a boundary correction term. \end{prop*}
In \cite{Ga} Gathmann discusses a theoretical algorithm for computing $(X|Y)$-relative and $Y$-restricted invariants inductively, exploiting the recursive structure of the boundary terms in the above formula. He then takes action in \cite{Ga-MF}, where he gives a different proof of Givental's mirror theorem based on this algorithm. Similarly, we are able to prove a quantum Lefschetz-type relation for some generating function of $2$-pointed quasimap invariants with one arbitrary descendant and one fundamental class insertion; see Theorem \ref{Theorem Quantum Lefschetz} and the discussion preceding it for notation and assumptions.
\begin{thm*} 
Let $X$ be a smooth projective variety, and $Y$ a smooth very ample divisor with $-K_Y$ nef, and containing all curve classes. Then
\begin{equation}\label{eqn:mirror}
\dfrac{\sum_{\beta\geq 0} q^\beta\prod_{j=0}^{Y\cdot\beta}(Y+jz)S_0^X(z,\beta)}{P_0^X(q)}= \tilde{S}_0^Y(z,q)
\end{equation}
where:
\begin{align*}
 P_0^X(q) % & = 1 + \sum_{\substack{\beta>0 \\ K_Y\cdot\beta=0}} q^\beta (Y\cdot\beta) \langle [\pt_Y],\mathbbm 1_{X}\rangle_{0,(Y\cdot\beta,0),\beta}^{X|Y} \\
            & = 1 + \sum_{\substack{\beta>0 \\ K_Y\cdot\beta=0}} q^\beta(Y\cdot\beta)!\langle [\pt_X] \psi_1^{Y\cdot\beta-1} ,\mathbbm 1_{X}\rangle_{0,2,\beta}^X
\end{align*}
\end{thm*}
\section*{Notation and conventions}
\addcontentsline{toc}{section}{Notations and conventions}
I work over an algebraically closed field $\kk$ of characteristic $0$. $\MM_{g,n}$ (possibly with decorations) denotes the Artin stack of prestable (geometrically connected, geometrically reduced, projective, flat and finitely presented morphisms of relative dimension $1$, with at most nodes as singularities) curves of genus $g$ with $n$ markings (smooth and disjoint sections); $\mathfrak{Pic}_{g,n}$ denotes the Picard stack of the universal curve $\mathcal C_{g,n}\to\MM_{g,n}$. I will normally use $C,\mathcal C,\plC$ for a curve, (universal) family of curves, and tropical curve respectively. They are usually stable with respect to some extra structure (e.g. a weight or a map), while $\widetilde{C}$ and $\overline{C}$ shall denote resp. a semistable modification and a contraction admitting a worse than nodal singularity. $\M{g}{n}{X}{\beta}$ denotes Kontsevich's moduli space of stable maps from a genus $g$, $n$-marked prestable curve to a smooth projective variety $X$ of degree $\beta\in\operatorname{Eff(X)}=H^2_+(X)$. In this setting $\pi$ will denote the universal curve, and $f$ the universal stable map (possibly with similar tilde and bar decorations as above).
