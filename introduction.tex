

\chapter*{Introduction}
\addcontentsline{toc}{chapter}{Introduction}
Kontsevich's moduli space of stable maps $\M{g}{n}{X}{\beta}$ \cite{KON} is meant to provide a first approximated answer to the question: How many smooth curves of genus $g$ and curve class $\beta$ are there in $X$, that pass through a number of subvarieties $V_1,\ldots,V_n\subseteq X$? Except in the $g=0$, $X$ homogeneous case (and few others), it turns out the geometry of $\M{g}{n}{X}{\beta}$ is overwhelmingly complicated (indeed these spaces satisfy Vakil-Murphy's law in algebraic geometry \cite{VakilMurphy}). Even though its properness makes it well-suited for intersection-theoretic computations, this space is often not endowed with a fundamental class (it may not be equidimensional). Rather it has a virtual fundamental class, that enjoys a number of properties expected from an actual fundamental class - if there existed one -, most notably deformation invariance, but is constructed by means of a somewhat heavy technical machinery that globalises the (expected) deformation theory of this moduli problem \cites{LiTian,BF,Behrend}. Even so, the ideal situation of a smooth curve embedding in $X$ is far from generic, both in the actual and virtual sense. Understanding these \emph{degenerate contributions} from curves of lower genus or degree has been one of the main difficulties since the origins of the subject, together with the similarly hard problem of actually computing the Gromov-Witten invariants: in most cases the \emph{computations} are relegated to some - clever but partial - generating functions, and the remarkable algebraic structures underlying them (quantum cohomology and Frobenius manifolds, see e.g. \cite{ManinFrobenius}), for a direct understanding of the moduli space of maps becomes virtually impossible as soon as the degree gets larger, or the target does not enjoy a clear modular interpretation itself. Comparing Kontsevich's moduli space of stable maps with different modular compactifications of the locus of maps from a smooth curve has proven both a beautiful and a fruitful technique, and it is this approach - in some simple instances - that I am going to discuss within this thesis.

Chapter \ref{ch:qm} deals with the theory of quasimaps. Quasimaps are defined whenever the target belongs to a large class of GIT quotients (when there are no strictly semistable points, the stabilisers are all trvial, and the space acted upon is at most mildly singular; see \cite{CFKM}), in particular when $X$ is a smooth projective toric variety \cite{CF-K}. Quasimaps can be thought of as maps to the quotient stack, such that the preimage of the unstable locus is finite and non-special. They are beloved by the algebraic geometer because their stronger stability condition forces the source curve not to have rational tails, nor contracted rational bridges. In turn, the stability condition can be made to depend on a positive rational parameter $\epsilon$, endowing $\mathbb Q_{>0}$ with a finite wall-and-chamber structure, such that the $\epsilon\gg0$ condition recovers Kontsevich's stable maps. Elegant wall-crossing formulae for the invariants (or rather their generating series) have been studied in \cites{TodaStableQuotient,CF-K-wallcrossing,CF-K-MirrorSymmetry,CF-K-higher-genus,CladerJandaRuan}. In \S \ref{sec:qm_review} I review and expand on some foundational material in the theory of quasimaps, comparing with well-known results in Gromov-Witten theory, and paying particular attention to the quasimap quantum product - whose relation with Batyrev's quantum ring is not so direct as I hoped in the strictly semi-positive case, see Section \ref{sec:Batyrev}. In \S \ref{sec:loc_formula} I review the localisation formula for toric quasimaps, which not only is a key ingredient in the proof of wall-crossing formulae for toric targets, but also can be made completely explicit in terms of weights for the big torus action, so that the computation of the invariants in some simple situations becomes a combinatorial game. Finally, in \S \ref{sec:rel_qm} I discuss the bulk of my joint work with N. Nabijou \cite{BN}, in which we introduce the notion of \emph{relative quasimap invariants} under the assumptions that $g=0$ and the target $(X|Y)$ is a smooth and very ample pair. We extend Gathmann's formula \cite{Ga} to this setting, relating the virtual classes of the relative quasimap spaces when the tangency requirement is increased by $1$; see Proposition \ref{prop:quasimap_Gathmann_formula}:
\begin{prop*} In the Chow group of $\Q{0}{\alpha}{X|Y}{\beta}$
 \begin{equation*} (\alpha_k \psi_k + x_k^*c_1(L_Y)) \cap \virt{\Q{0}{\alpha}{X|Y}{\beta}} = \virt{\Q{0}{\alpha+e_k}{X|Y}{\beta}} + \virt{\mathcal{D}^Q_{\alpha,k}(X,\beta)} \end{equation*}
where $\mathcal{D}^Q_{\alpha,k}(X,\beta)$ is a boundary correction term. \end{prop*}
In \cite{Ga} Gathmann discusses a theoretical algorithm for computing $(X|Y)$-relative and $Y$-restricted invariants inductively, exploiting the recursive structure of the boundary terms in the above formula. He then takes action in \cite{Ga-MF}, where he gives a different proof of Givental's mirror theorem based on this algorithm. Analogously, we are able to prove a quantum Lefschetz-type relation for some generating function of $2$-pointed quasimap invariants with one arbitrary descendant and one fundamental class insertion, mimicking Givental's $J$-function in Gromov-Witten theory; see Theorem \ref{Theorem Quantum Lefschetz} and the discussion preceding it for notation and assumptions.
\begin{thm*}[Quasimap quantum Lefschetz]
Let $X$ be a smooth projective toric variety, $Y$ a smooth very ample divisor with $-K_Y$ nef, and such that it contains all the curve classes. Then
\begin{equation*}
\dfrac{\sum_{\beta\geq 0} q^\beta\prod_{j=0}^{Y\cdot\beta}(Y+jz)S_0^X(z,\beta)}{P_0^X(q)}= \tilde{S}_0^Y(z,q)
\end{equation*}
where:
\begin{align*}
 P_0^X(q) % & = 1 + \sum_{\substack{\beta>0 \\ K_Y\cdot\beta=0}} q^\beta (Y\cdot\beta) \langle [\pt_Y],\mathbbm 1_{X}\rangle_{0,(Y\cdot\beta,0),\beta}^{X|Y} \\
            & = 1 + \sum_{\substack{\beta>0 \\ K_Y\cdot\beta=0}} q^\beta(Y\cdot\beta)!\langle [\pt_X] \psi_1^{Y\cdot\beta-1} ,\mathbbm 1_{X}\rangle_{0,2,\beta}^X
\end{align*}
\end{thm*}

\smallskip

In Chapter \ref{ch:hg} I discuss the genus one Gromov-Witten theory of projective hypersurfaces, especially the quintic threefold. In the genus zero case, $\M{0}{n}{\PP^N}{d}$ is a smooth stack, and there is a vector bundle $\pi_*f^*\OO_{\PP^N}(l)$ with an induced section $\tilde{s}$ (virtually) cutting out the locus of maps to $X_l=V(s)\subseteq \PP^N$ \cite{CKL}. Therefore the restricted genus zero Gromov-Witten invariants of a hypersurface (or, more generally, a complete intersection) can be computed as \emph{twisted} invariants of projective space. The higher genus situation is more intricate: $\M{g}{n}{\PP^N}{d}$ has many components of different dimensions, and $\pi_*f^*\OO_{\PP^N}(l)$ is only a sheaf. The torus localised theory twisted by $c_{\rm{top}}(\R\pi_*f^*\OO_{\PP^4}(5))$ goes under the name of \emph{formal quintic}: it is different from - but non-trivially related to - the theory of the quintic, and it has recently received considerable interest \cites{LhoPandha,GJR}. We consider the genus one case in detail, whose geometry is far better understood than higher genus, mostly thanks to work or R. Vakil and A. Zinger \cites{Vre,VZpreview}: both the boundary components and the smoothable elements are known. Two apparently opposite approaches have emerged to reduce the complexity of this problem: the first one, developed in a series of papers by J. Li, R. Vakil, and A. Zinger, consists in a desingularisation of the main component of $\M{1}{n}{\PP^N}{d}$ via an iterated blow-up procedure along boundary loci \cites{VZ,HL}, so that on the resulting space $\VZ{n}{\PP^N}{d}$ there is a vector bundle $\tilde{\pi}_*\tilde{f}^*\OO_{\PP^N}(l)$ that can be used to define \emph{reduced} invariants of a complete intersection. By construction these invariants capture only the contribution of the main component, and are therefore expected to have a more basic enumerative content. Indeed the relation with ordinary genus one Gromov-Witten invariants has been studied in \cites{LZ,zingerstvsred} (in the symplectic category): they turn out to differ by an explicit genus zero correction term. The simplest formulation is that for the quintic threefold $X_5\subseteq\PP^4$, which has also been proved with algebro-geometric techniques by J. Li and H.L. Chang \cite{CL}:
\begin{equation}\tag{Li-Zinger formula for the quintic threefold}
 N_{1,d}(X_5)=N_{1,d}^{\rm{red}}(X_5)+\frac{1}{12}N_{0,d}(X_5)
\end{equation}
In \S \ref{sec:redinv} I describe the structure of $\M{1}{n}{\PP^N}{d}$, I review the Vakil-Zinger blow-up procedure (following Y. Hu and J.Li), and I discuss to some extent the Li-Zinger formula for projective spaces of small dimension, which is particularly easy because we understand the geometry of the moduli spaces of maps in this case.

The second approach is due to M. Viscardi \cite{VISC}, building on previous work of D.I. Smyth on Gorenstein singularities of genus one and alternate compactifications of the space of smooth pointed elliptic curves \cite{SMY1}. The idea is that allowing (smoothable) maps from more singular than nodal curves, and at the same time making their stable models unstable, one is able to define alternative compactifications of the space of maps from a smooth elliptic curve, that have less boundary components than ordinary stable maps, and ultimately become irreducible when the target is $\PP^N$, and the integer $m$ controlling the permitted singularities is large enough compared to the other numerical invariants.

The punchline is that the two approaches are intimately related. In \S \ref{sec:singularities} I review Smyth's singularities, describing a number of useful properties of theirs, and Viscardi's moduli spaces of maps; I prove the existence of the $1$-stabilisation morphism at the level of weighted-stable curves (from my paper with F. Carocci and C. Manolache \cite{BCM}), and start a discussion of the more complicated genus two geometry (also from discussions with F. Carocci). I explain how aligned log structures and the factorisation through a Smyth's singularity allowed the authors of \cite{RSPW} to provide a modular interpretation for the Vakil-Zinger's desingularisation. Finally, I state the main result of \cite{BCM}, equating the reduced invariants of the quintic threefold with those arising from Viscardi's space of maps from pseudostable \cite{Schubert} (i.e. admitting only nodes and cusps as singularities) curves, see Theorem \ref{thm:redvscusp}:
\begin{thm*}
 For a smooth quintic threefold $X_5\subseteq \PP^4$, \[GW_1^{\mathrm{red}}(X_5)=GW_1^{\mathrm{cusp}}(X_5).\]
\end{thm*}
Our proof is by an extension of techniques due to H.L. Chang, Y. Hu, Y.H. Kiem, and J. Li to the present situation, which I outline in the technical section \ref{sec:techniques}.

In \S \ref{sec:relativeone} I outline an ongoing project with N. Nabijou and D. Ranganathan: the goal is to apply Gathmann's techniques to the Vakil-Zinger's desingularisation, in order to define \emph{reduced relative} invariants in genus one, and obtain a quantum Lefschetz-type result for reduced invariants, the scope of which will extend Zinger's computation for Calabi-Yau hypersurfaces \cite{Zinger-CYhyp}. It seems a favourable moment for such a study, thanks to the new light shed on this topic by Ranganathan--Santos-Parker--Wise's beautiful work \cites{RSPW,RSPW2}. This section is admittedly vague, fuzzy and incomplete, but my enthusiasm for this project could not be held and poured in a hopefully useful discussion.

As a last word, I would like to clarify that most of the mathematics discussed within this thesis originated from my collaborations with F. Carocci, C. Manolache, N. Nabijou, and D. Ranganathan; I have learnt much working with them, and I gratefully acknowledge this, while the exposition of the material presented in this thesis (as well as any mistake it can possibly contain) ought to be ascribed to me only.

\section*{Notation and conventions}
\addcontentsline{toc}{section}{Notations and conventions}
I work over an algebraically closed field $\kk$ of characteristic $0$. $k[\epsilon]\simeq k[t]/(t^2)$ will denote the ring of dual numbers. $[n]$ will denote the set of natural numbers $\{1,\ldots,n\}$. $\MM_{g,n}$ (possibly with decorations) denotes the Artin stack of prestable curves (geometrically connected, geometrically reduced, projective, flat and finitely presented morphisms of relative dimension $1$, with at most nodes as singularities of the fibers) of genus $g$ with $n$ markings (smooth and disjoint sections); $\mathfrak{Pic}_{g,n}$ denotes the Picard stack of the universal curve $\mathcal C_{g,n}\to\MM_{g,n}$. I will normally use $C,\mathcal C,\plC$ for a curve, (universal) family of curves, and tropical curve respectively, $\Gamma(C)$ for the dual graph of $C$; a subcurve is always connected. They are usually stable with respect to some extra structure (e.g. a weight or a map), while $\widetilde{C}$ and $\overline{C}$ shall denote respectively a semistable modification, and a contraction admitting a worse than nodal singularity. $\M{g}{n}{X}{\beta}$ denotes Kontsevich's moduli space of stable maps from a genus $g$, $n$-marked prestable curve to a smooth projective variety $X$ of degree $\beta\in\operatorname{Eff(X)}=H^2_+(X)$. In this setting $\pi$ will denote the universal curve, and $f$ the universal stable map (possibly with similar tilde and bar decorations as above), the universal sections (markings) will be denoted either by $x$ or by $p$, while $\ev_k=f\circ x_k\colon \M{g}{n}{X}{\beta}\to X$; $q$ will mostly be used for a separating node.
