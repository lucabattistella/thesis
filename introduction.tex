

\chapter*{Introduction}
\addcontentsline{toc}{chapter}{Introduction}
Kontsevich's moduli space of stable maps $\M{g}{n}{X}{\beta}$ \cite{KON} is meant to provide a first approximated answer to the question: How many smooth curves of genus $g$ and curve class $\beta$ are there in $X$, that pass through a number of subvarieties $V_1,\ldots,V_n\subseteq X$? Except in the $g=0$, $X$ homogeneous case (and few others), it turns out the geometry of $\M{g}{n}{X}{\beta}$ is overwhelmingly complicated (indeed these spaces satisfy Vakil-Murphy's law in algebraic geometry \cite{VakilMurphy}). Even though its properness makes it well-suited for intersection-theoretic computations, this space is often not endowed with a fundamental class (it may not be equidimensional), but rather with a virtual fundamental class, that enjoys a number of properties expected from an actual fundamental class if there existed one (most notably deformation invariance), but is constructed by means of a rather heavy technical machinery that globalises the (expected) deformation theory of this moduli problem \cites{LiTian,BF}. Even so, the ideal situation of a smooth curve embedding in $X$ is far from generic, both in the actual and virtual sense. Understanding these degenerate contributions from curves of lower genus or degree has been one of the main difficulties since the origins of the subject, in parallel with the equally hard problem of actually computing the Gromov-Witten invariants, for which it has been necessary to introduce clever generating series and to study the remarkable algebraic structures underlying them (quantum cohomology and Frobenius manifolds, see e.g. \cite{ManinFrobenius}), for a direct understanding of the moduli space of maps becomes virtually impossible as soon as the degree gets larger. Comparing Kontsevich's moduli space of stable maps with different modular compactifications of the locus of maps from a smooth curve has proven both a beautiful and a fruitful technique, and it is this approach -in some simple instances- that I am going to discuss within this thesis.


\section*{Notation and conventions}
\addcontentsline{toc}{section}{Notations and conventions}
I work over an algebraically closed field $\kk$ of characteristic $0$. $\MM_{g,n}$ (possibly with decorations) denotes the Artin stack of prestable (geometrically connected, geometrically reduced, projective, flat and finitely presented morphisms of relative dimension $1$, with at most nodes as singularities) curves of genus $g$ with $n$ markings (smooth and disjoint sections); $\mathfrak{Pic}_{g,n}$ denotes the Picard stack of the universal curve $\mathcal C_{g,n}\to\MM_{g,n}$. I will normally use $C,\mathcal C,\plC$ for a curve, (universal) family of curves, and tropical curve respectively. They are usually stable with respect to some extra structure (e.g. a weight or a map), while $\widetilde{C}$ and $\overline{C}$ shall denote resp. a semistable modification and a contraction admitting a worse than nodal singularity. $\M{g}{n}{X}{\beta}$ denotes Kontsevich's moduli space of stable maps from a genus $g$, $n$-marked prestable curve to a smooth projective variety $X$ of degree $\beta\in\operatorname{Eff(X)}=H^2_+(X)$. In this setting $\pi$ will denote the universal curve, and $f$ the universal stable map (possibly with similar tilde and bar decorations as above).
